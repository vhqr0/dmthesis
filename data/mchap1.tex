\chapter{引言}
\label{cha:introduction}

\section{研究背景}

IPv4 网络扫描是简单的. IPv4 $2^{32}$ 的地址空间对现代计算机而言并非庞大, 使用了无状态技术的网络扫描工具如 Zmap, masscan 等可以在几个小时内扫描整个 IPv4 网络. 狭小的地址空间也限制了 IPv4 地址的配置方式: 网络管理员们试图充分利用每一个公网地址, 即使使用 NAT 技术的私有地址, 狭小的主机号地址空间也多由手动分配或使用 DHCP 连续分配. 这些因素都使得网络扫描工具如 nmap 可以快速扫描一个网段.

但 IPv6 则不同. IPv6 有巨大的 $2^{128}$ 的地址空间, 甚至 ISP 默认提供 $2^{96}$ 的地址空间. 凭借于此, IPv6 地址配置策略也多种多样: 除了继承自 IPv4 的手动分配和 DHCP, 常见的地址配置方式还有 EUI-64 和随机地址. 除此之外, 新的更加复杂的地址配置方式也被不断提出. 如充分利用 IPv6 地址空间, DHCP 每次分配单向加密过的无规律可循的 IPv6 地址. 这些因素使得以往在 IPv4 网络中常用的扫描工具对 IPv6 网络不再有效. 因此, 我们期望找到一个新的方法来扫描 IPv6 网络.

\section{研究方法概述}

IPv6 网络扫描的难点在于其巨大的地址空间. 本文的主要工作是提出了一种 IPv6 地址生成模型, 该模型通过对一组 IPv6 地址的学习, 了解 IPv6 地址空间结构, 进而生成 IPv6 地址. 通过该模型, 我们可以缩小 IPv6 地址空间, 提高 IPv6 网络扫描的效率.

该模型主要基于以下三个技术:

\begin{itemize}
\item 信息熵分析: 本文使用 IPv6 地址 32 个半字节的信息熵来判断其变动率, 通过比较相邻半字节的信息熵来对 IPv6 地址进行分段, 当相邻的半字节信息熵相近时, 便将其合并为一个更大的段. 例如, 数据集中如果存在大量遵循 EUI-64 规范的 IPv6 地址会降低第 23-26 个半字节的信息熵, 从而使 23-26 个半字节被合并为一段.
\item DBSCAN: 本文使用的第二个技术是基于 DBSCAN 算法对数据集进行无监督机器学习, 对基于信息熵分析划分的每个段的值进行聚类.
\item 贝叶斯网络: 本文使用的第三个技术是基于贝叶斯网络对数据集进行统计建模, 以一种分层的方式自动确定段值集群中的条件概率. 通过机器学习的方式发现基于信息熵的特征揭示 IPv6 地址空间结构要优于研究人员视觉检查大量 IPv6 地址.
\end{itemize}

该模型以循序渐进的方式, 首先收集数据集, 通过信息熵分析分段, 通过 DBSCAN 对段值聚类, 最后建立贝叶斯网络模型.

当扫描一个 IPv6 网络时, 我们不必扫描其内所有地址, 而是扫描通过该模型生成的若干地址. 这种扫描的准确性受制于数据集对要扫描的网络的代表性.

\section{相关工作}

早在 2004 年, Strayer 等人~\cite{ref:strayer}~就将信息熵分析应用到 IPv6 地址中. 他们注意到, IPv6 地址每字节的信息熵要原低于 IPv4, 这很好理解: 在 IPv6 地址中, 信息不必像 IPv4 地址那样压缩到 4 字节中, 相反, 信息可以在 IPv6 地址中传播. 从那时其起, IPv6 地址隐私扩展的引入和普及改变了这种状况, 它为 IPv6 地址增加了随机值, 从而抬高了信息熵. 虽然随机值有较高的信息量, 但都是一些不相关的网络结构. Malone~\cite{ref:malone}~, Gont 和 Chown~\cite{ref:gont}~, Plonka 和 Berger~\cite{ref:plonka}~ 等人的工作都尝试识别 IPv6 地址中的伪随机数, 本文也是如此, 不同的是本文利用了信息熵.