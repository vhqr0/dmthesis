\begin{cabstract}

  为解决 IPv6 网络扫描问题,本文提出了一种基于信息熵, DBSCAN, 贝叶斯网络等算法的 IPv6 地址生成模型. 该模型通过学习 IPv6 地址空间结构特征生成 IPv6 地址, 有效提高了 IPv6 网络扫描的效率.

  本文的创新点主要有:

  \begin{itemize}
  \item 基于半字节的信息熵进行段划分;
  \item 基于 DBSCAN 进行段分类;
  \item 基于贝叶斯网络进行关联性分析.
  \end{itemize}

\end{cabstract}

\ckeywords{IPv6; 信息熵; DBSCAN; 贝叶斯网络}

\begin{eabstract}

  In order to solve the problem of IPv6 network scanning, this paper proposes an IPv6 address generation model based on information entropy, DBSCAN, Bayesian network and other algorithms. This model generates IPv6 addresses by learning the IPv6 address space structure, which effectively improves the efficiency of IPv6 network scanning.

  The main innovations of this article are:

  \begin{itemize}
  \item Segmentation based on nibble information entropy;
  \item Segment classification based on DBSCAN;
  \item Relevance analysis based on Bayesian network.
  \end{itemize}

\end{eabstract}

\ekeywords{IPv6; Information entropy; DBSCAN; Bayesian network}
